%%=============================================================================
%% Samenvatting
%%=============================================================================

% TODO: De "abstract" of samenvatting is een kernachtige (~ 1 blz. voor een
% thesis) synthese van het document.
%
% Deze aspecten moeten zeker aan bod komen:
% - Context: waarom is dit werk belangrijk?
% - Nood: waarom moest dit onderzocht worden?
% - Taak: wat heb je precies gedaan?
% - Object: wat staat in dit document geschreven?
% - Resultaat: wat was het resultaat?
% - Conclusie: wat is/zijn de belangrijkste conclusie(s)?
% - Perspectief: blijven er nog vragen open die in de toekomst nog kunnen
%    onderzocht worden? Wat is een mogelijk vervolg voor jouw onderzoek?
%
% LET OP! Een samenvatting is GEEN voorwoord!

%%---------- Nederlandse samenvatting -----------------------------------------

\IfLanguageName{english}{%
\selectlanguage{dutch}
\chapter*{Samenvatting}
\selectlanguage{english}
}{}

%%---------- Samenvatting -----------------------------------------------------
% De samenvatting in de hoofdtaal van het document

\chapter*{Samenvatting}

Alhoewel meer en meer processen wereldwijd volledig digitaal plaatsvinden, worden toch nog een grote deel van procedures en data opslag uitgevoerd op niet-digitale manieren. Zo krijgen de meeste mensen hun factures nog steeds per brief, kassatickets worden nog steeds afgedrukt op papier, notities nemen op papier blijft de populaire keuze hoewel er tal van notitie-apps bestaan, etc. Dit heeft tot gevolg dat essentiële data nog massaal op een niet-digitale media bewaard wordt, namelijk op papier.

Tot enkele jaren geleden was dit probleem niet zo beduidend maar nu meer digitale platformen voor dataverwerking gebruikt worden, is het omzetten van data op papier naar digitale data, m.a.w. het digitalisatieproces steeds belangrijker geworden. 

Hierdoor werden tal van digitalisatiesofwareproducten ontwikkeld, zoals Abby FineReader en Adobe Acrobat Pro DC. Hoewel deze software producten veel features hebben, zijn ze betalend en closed source. Toch hebben enkele bedrijven enkele van hun digitalisatie oplossingen open source gemaakt, zoals Google met diens bekende OCR-software, Tesseract OCR, die door iedereen gebruikt kan worden om tekst in foto's om te zetten in tekstdata. 



